% !TeX root = ../main.tex

\chapter{引言}

\section{研究背景与意义}          % 介绍大模型的发展及其资源需求
第一段:介绍人工智能与大模型的发展背景,突出大模型在各领域的成功应用及其重要性。
第二段:阐述大模型在训练和推理阶段对计算、存储、通信和能耗等资源的高需求,强调资源瓶颈对大模型发展的制约作用。
第三段:说明提升资源利用效率对于推动大模型发展的重要意义,提出本文的研究动机。
需要提炼大模型计算的几大特征:分布式计算,异构性资源,动态负载。

\section{研究挑战}              % 现有工作的问题
1. 大模型推理计算时的能耗巨大,而目前的能效优化方法存在局限性,难以适应动态负载变化,导致资源浪费严重。
2. 大模型训练中的通信开销随着模型规模和上下文长度的增加而爆炸,传统的并行方案难以有效缓解这一问题,影响整体训练效率。
3. 异构资源环境下的并行策略设计复杂,依赖人工经验,难以适应多维资源约束,限制了资源利用效率的提升。

\section{本文的研究思路与关键问题}  % 本文思路会遇到的问题
针对上述挑战,本文提出了以下研究思路:
1. 设计动态能效优化策略,结合硬件特性和负载变化,实现推理阶段的能耗降低。
2. 提出新型并行范式,减少训练阶段的通信开销,提高训练吞吐量。
3. 构建自动化并行策略发现框架,适应异构资源环境,实现高效资源利用。
\section{本文的主要贡献}
\subsection{针对大模型高能效推理的能效优化}
\subsection{针对大模型高能效推理的能效优化}
\subsection{针对大模型高能效推理的能效优化}

\section{本文的组织结构}
本文共分为六章,结构安排如下:

第一章介绍本研究的背景与意义。

第二章介绍研究的基础背景与相关工作。

第三章介绍了本研究在对于高能效进行大模型推理的优化方法。

第四章介绍了本研究对于长文本大模型训练的通信优化方法。

第五章介绍了本研究在并行策略自动发现方面的工作。

第六章是全文的总结以及对未来方向的一些探讨。